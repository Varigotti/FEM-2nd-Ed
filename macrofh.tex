\newcommand{\bild}{2}
\newcommand {\bite}   {\begin {itemize}}         % bite
\newcommand {\eite}   {\end {itemize}}           % eite
\newcommand{\vek}[1]{\mbox{\boldmath $#1$}}
\newenvironment{Eqnarray} {\arraycolsep 0.14em\begin{eqnarray}}{\end{eqnarray}}
\newenvironment{EqnarrayNN} {\arraycolsep 0.14em\begin{eqnarray*}}{\end{eqnarray*}}
\newcommand {\bfo}    {\begin {Eqnarray}}       % bfo
\newcommand {\efo}    {\end {Eqnarray}}         % efo
\newcommand {\bfoo}    {\begin {EqnarrayNN}}       % bfo
\newcommand {\efoo}    {\end {EqnarrayNN}}         % efo
\newcommand {\nn} {\nonumber}
\newcommand {\dotprod}{{\,\scriptscriptstyle \stackrel{\bullet}{{}}}\,}
\newcommand{\FEM}{Methode der finiten Elemente\ }
\newcommand{\REM}{Methode der Randelemente\ }
\newcommand{\PV}{Prinzip der virtuellen Verr"uckungen\ }
\newcommand{\PVK}{Prinzip der virtuellen Verr"uckungen}
\newcommand {\barr}   {\begin {array}}           % barr
\newcommand {\earr}   {\end {array}}             % earr
\newcommand{\il}{\int_0^{\,l}}
\newcommand{\ig}{\int_\Gamma}
\newcommand{\io}{\int_\Omega}
%\newcommand{\do}{d\Omega}
\newcommand {\bquo}   {\begin {quote}}           % bquo
\newcommand {\equo}   {\end {quote}}             % equo
\newcommand {\queq}  [1]{(\ref{#1})}             % queq
\newcommand{\Np}{\varphi}
\def\strut{\rule{0in}{.50in}}
\newcommand {\absatz} \\
\newcommand {\dgl} {Differentialgleichung}
\newcommand{\hlq}{\glq\kern.07em\allowhyphens}   % Frank Holzwarth  Januar 2001
\newcommand{\lqq}{\lq\lq}
\newcommand{\rqq}{\rq\rq \,}
\newcommand{\beq}{\begin{equation}}
\newcommand{\eeq}{\end{equation}}
